\documentclass[11pt,a4paper]{moderncv}

\moderncvtheme[blue]{classic}
\usepackage{hyperref}
\usepackage{fontspec}
\usepackage{xunicode}
\usepackage{xeCJK}
\setmainfont{Helvetica}
\setsansfont{Helvetica}
\setmonofont{Courier New}
\setCJKmainfont{STSong}
\setCJKsansfont{STKaiti}
\setCJKmonofont{Hiragino Sans GB W3}

\usepackage{CJKutf8}
\usepackage[utf8]{inputenc}
\usepackage[scale=0.8]{geometry}

\usepackage{hyperref}
\definecolor{linkcolour}{rgb}{0,0.2,0.6}
\hypersetup{colorlinks,breaklinks,urlcolor=linkcolour, linkcolor=linkcolour}

\renewcommand*{\cventry}[7][.25em]{
  \cvitem[#1]{#2}{
    {\bfseries#3}
    \ifthenelse{\equal{#4}{}}{}{,{\slshape#4}}
    \ifthenelse{\equal{#5}{}}{}{,#5}
    \ifthenelse{\equal{#6}{}}{}{,#6}
    。\strut
    \ifx&#7&
      \else{\newline{}\begin{minipage}[t]{\linewidth}\small#7\end{minipage}}\fi}}

\firstname{聂伟琳}
\title{后端工程师}
\mobile{+86 185-6563-5571}
\email{winiex.nie@gmail.com}
\homepage{bachiscoding.com}

\makeatletter
\renewcommand*{\bibliographyitemlabel}{\@biblabel{\arabic{enumiv}}}
\makeatother

\begin{document}
\maketitle

\section{个人信息}
%\cventry{year--year}{Degree}{Institution}{City}{\textit{Grade}}{Description}
\cvline{性别}{男}
\cvline{籍贯}{湖北\ 潜江}
\cvline{出生年月}{1992 年 3 月}
\cvline{教育经历}{华中科技大学\ 计算机科学与技术专业\ 学士}
\cvline{特点}{拥有较强的快速学习能力,期待用技术提供有价值的服务}

\section{工作履历}
\cventry{2013 年 4 月 - 2015 年 3 月}{后端研发工程师(Python)、项目最早期参与者}{恋爱记}{武汉滴滴网络科技有限公司}{}{主要职责为:后端系统从零开始的技术选型、方案设计以及最终实现;新需求的研发和既有代码的维护;技术人才的招聘、面试以及培养;产品层面的技术咨询和公司层面的决策。项目于 2014 年底获得 150 万美元风险投资。\newline{}}
\cventry{2015 年 3 月 - 至今}{后端数据平台高级工程师(Java、Hadoop、Python)}{爱美购}{深圳爱美购网络科技有限公司}{}{主要职责为:早期参与公司后端服务程序的研发、维护,后至今负责公司大数据平台的基础设施构建、维护以及基于其的数据产品的研发、维护,为公司产品研发、内容运营、营收等重要决策过程提供稳定的数据支持服务。\newline{}}

\section{项目经历}

\subsection{武汉滴滴网络科技有限公司期间}
\cvline{恋爱记}{参与了恋爱记服务端程序从零开始的设计、研发、维护。主要使用 Python 语言进行开发,并基于 MySQL、Redis、Celery、Django 等基础软件搭建了多机器后端系统,能够支持一定的并发量,高峰期日处理动态请求数可达千万次。对外设计并实现了 RESTFul HTTP 接口供客户端使用,对内设计并实现了基于 Thrift 的服务接口供其它服务使用,进而确立了后端服务化的方向。实现了内容同步算法保持多用户、多客户端之间的内容同步。}
\cvline{爱芽社区}{参与了恋爱记应用内爱芽社区服务端程序的设计、研发、维护。技术选型同恋爱记。}

\subsection{深圳爱美购网络科技有限公司期间}
\cvline{后端业务系统}{参与了爱美购后端程序 Shopping 的新特性研发以及维护,进而了解了海淘领域复杂的业务逻辑。}
\cvline{数据基础系统第一版(UBT、离线计算)}{使用 Cloudera 搭建了大数据基础环境;设计了以有限状态自动机为基础模型的数据规范,实现了其解析程序、入库程序以及基于 MapReduce 计算框架的离线分析程序;基于最终产出的数据实现了简单的报表系统后端服务;数据主要由客户端打点上报。}
\cvline{数据基础系统第二版(类 Dremel 系统)}{使用 Cloudera 重新搭建了大数据基础平台;为基于 Go 语言实现的日志收集工具 Heka 编写了业务定制化的插件并部署至服务端;编写了从 Kafka 多分区并行导入原始数据至 HDFS 的工具;根据需求实现了从原始数据分离出结构化的、业务友好的数据的 ETL 程序;一部分热点数据使用 Parquet 列式存储格式存储,并提供给 Impala、PrestoDB 等前端 SQL 
执行引擎使用,达到了秒级别的查询效率;实现了日志申请、审批、入库的自动化流程,达到了日志打点自动化的目的;数据主要由服务端上报,客户端辅助。}
\cvline{数据运营支持系统}{基于 PrestoDB 提供的 SQL 查询接口实现了 Dashboard 报表系统的后端服务;构建了以数据为核心参考指标的选品、内容运营支持系统。}

\section{主要技能}
\cvline{技术经验}{5 年,2010 年至今}
\cvline{编程语言}{Python(熟练),Java(掌握),Go(了解)}
\cvline{操作系统}{Mac OS X、Linux(ArchLinux、Ubuntu、CentOS)(熟练,能使用、运维及自动化)}
\cvline{开发工具}{Vim(熟练),Git \& Git Workflow(掌握),Shell(Bash)(了解)}
\cvline{基础软件}{MySQL、Redis、Hadoop(掌握),Nginx、PrestoDB、Impala、Hive、Cloudera、Hue(了解)}
\cvline{框架}{Flask、Tornado、MapReduce 计算框架(掌握,Tornado 有阅读、研究一部分源码)}
\cvline{计算机科学}{算法和数据结构(掌握),计算机网络、编译原理和编程语言原理、计算机系统及硬件体系、计算机操作系统(了解)}
\cvline{文档撰写}{Markdown(熟练),Word、PRD 文档及脑图(了解)}
\cvline{外语}{英语(通过六级,流畅地读,较流畅地听、说、写,常使用于技术社区内交流,日常沟通无障碍)}

\section{博客文章}
\cvline{【译注】Columnar Storage - 列式存储}{\href{http://bachiscoding.com/blog/2015/11/24/annotations-on-paper-trail-columnar-storage/}{http://bachiscoding.com/blog/2015/11/24/annotations-on-paper-trail-columnar-storage/}}
\cvline{关于快速排序算法的一些想法}{\href{http://bachiscoding.com/blog/2014/12/26/some-thoughts-on-quicksort/}{http://bachiscoding.com/blog/2014/12/26/some-thoughts-on-quicksort/}}

\section{其它信息}
\cvline{Github}{\href{https://github.com/winiex}{https://github.com/winiex}}
\cvline{About Me}{\href{http://about.me/winiex}{http://about.me/winiex}}
\cvline{兴趣爱好}{音乐、阅读}

\end{document}
