\documentclass[11pt,a4paper]{moderncv}

\moderncvtheme[blue]{classic}
\usepackage{fontspec}
\usepackage{xunicode}
\usepackage{xeCJK}
\setmainfont{Minion Pro}
\setsansfont{Myriad Pro}
\setmonofont{Courier New}
\setCJKmainfont{STSong}
\setCJKsansfont{STKaiti}
\setCJKmonofont{Hiragino Sans GB W3}

\usepackage{CJKutf8}
\usepackage[utf8]{inputenc}
\usepackage[scale=0.8]{geometry}

\usepackage{hyperref}
\definecolor{linkcolour}{rgb}{0,0.2,0.6}
\hypersetup{colorlinks,breaklinks,urlcolor=linkcolour, linkcolor=linkcolour}

\renewcommand*{\cventry}[7][.25em]{
  \cvitem[#1]{#2}{
    {\bfseries#3}
    \ifthenelse{\equal{#4}{}}{}{,{\slshape#4}}
    \ifthenelse{\equal{#5}{}}{}{,#5}
    \ifthenelse{\equal{#6}{}}{}{,#6}
    。\strut
    \ifx&#7&
      \else{\newline{}\begin{minipage}[t]{\linewidth}\small#7\end{minipage}}\fi}}

\firstname{聂伟琳}
\title{后端工程师}
\mobile{+86 185-6563-5571}
\email{winiex.nie@gmail.com}
\homepage{http://bachiscoding.com/}

\makeatletter
\renewcommand*{\bibliographyitemlabel}{\@biblabel{\arabic{enumiv}}}
\makeatother

\begin{document}
\maketitle

\section{个人信息}
%\cventry{year--year}{Degree}{Institution}{City}{\textit{Grade}}{Description}
\cvline{性别}{男}
\cvline{籍贯}{湖北\ 潜江}
\cvline{出生年月}{1992 年 3 月}
\cvline{教育经历}{华中科技大学\ 计算机科学与技术专业\ 学士}
\cvline{特点}{拥有较强的快速学习能力,期待用技术提供有价值的服务}

\section{工作履历}
\cventry{2013 年 7 月 - 2013 年 9 月}{后端研发实习生(PHP)}{美丽说}{北京美丽说网络科技有限公司}{}{主要职责为参与美丽说网站后端系统新特性的实现以及既有业务的维护。在熟悉了公司内部基于 PHP 的 Web 框架后于其基础之上进行开发,期间在导师指导下参与过后端服务核心模块的优化。\newline{}}
\cventry{2013 年 4 月 - 2015 年 3 月}{后端研发工程师(Python)、项目最早期参与者}{恋爱笔记}{武汉滴滴网络科技有限公司}{}{主要职责为:后端系统从零开始的技术选型、方案设计以及最终实现;新需求的研发和既有代码的维护;技术人才的招聘、面试以及培养;产品层面的技术咨询和公司层面的决策。\newline{}}
\cventry{2015 年 3 月 - 至今}{后端数据平台高级工程师(Java、Hadoop、Python)}{爱美购}{深圳爱美购网络科技有限公司}{}{早期参与公司后端服务程序的研发、维护,后至今负责公司大数据平台的基础设施构建、维护以及基于其的数据产品的研发、维护,为公司产品研发、营收等重要决策过程提供稳定的数据支持服务。\newline{}}

\section{项目经历}
\subsection{北京美丽说网络科技有限公司期间}
\cvline{美丽说后端业务系统}{参与了美丽说后端业务系统的研发、维护。与前端工程师一起完成了美丽说账户体系中雅虎邮箱用户的迁移工作,完成了美丽说热门单品下架检测程序,进行了开放平台分享程序的重构与维护。}

\subsection{武汉滴滴网络科技有限公司期间}
\cvline{恋爱记}{参与了恋爱记服务端程序从零开始的设计、研发、维护。主要使用 Python 语言进行开发,并基于 MySQL、Redis、Celery、Django 等基础软件搭建了多机器后端系统,能够支持一定的并发量,高峰期日处理动态请求数可达千万次。对外设计并实现了 RESTFul HTTP 接口供客户端使用,对内设计并实现了基于 Thrift 的服务接口供其它服务使用,进而确立了后端服务化的方向。实现了内容同步算法保持多用户、多客户端之间的内容同步。}
\cvline{爱芽社区}{参与了恋爱记应用内爱芽社区服务端程序的设计、研发、维护。技术选型同恋爱记。}

\subsection{深圳爱美购网络科技有限公司期间}
\cvline{后端业务系统}{参与了爱美购后端程序 Shopping 的新特性研发以及维护,进而了解了海淘领域复杂的业务逻辑。}
\cvline{数据基础系统第一版(UBT、离线计算)}{使用 Cloudera 搭建了大数据基础环境;设计了以有限状态自动机为基础模型的数据规范,实现了其解析程序、入库程序以及基于 MapReduce 计算框架的离线分析程序;基于最终产出的数据实现了简单的报表系统后端服务;数据主要由客户端打点上报。}
\cvline{数据基础系统第二版(类 Dremel 系统)}{使用 Cloudera 重新搭建了大数据基础平台;为基于 Go 语言实现的日志收集工具 Heka 编写了业务定制化的插件并部署至服务端;编写了从 Kafka 多分区并行导入原始数据至 HDFS 的工具;根据需求实现了从原始数据分离出结构化的、业务友好的数据的 ETL 程序;一部分热点数据使用 Parquet 列式存储格式存储,并提供给 Impala、PrestoDB 等前端 SQL 
执行引擎使用,达到了秒级别的查询效率;实现了日志申请、审批、入库的自动化流程,达到了日志打点自动化的目的;数据主要由服务端上报,客户端辅助。}
\cvline{数据运营支持系统}{基于 PrestoDB 提供的 SQL 查询接口实现了 Dashboard 报表系统的后端服务;构建了以数据为核心参考指标的选品、内容运营支持系统。}

\section{主要技能}
\cvline{技术年龄}{5 年,2010 年至今}
\cvline{编程语言}{Python、Java(掌握),Go(了解)}
\cvline{操作系统}{Mac OS X、Linux(ArchLinux、Ubuntu、CentOS)(使用、简单运维及自动化)、Windows(使用)}
\cvline{开发工具}{Vim、Git \& Git Workflow、GitHub \& BitBucket(熟悉),Shell(Bash)(使用)}
\cvline{基础软件}{MySQL、Redis、Hadoop(熟悉),Nginx、PrestoDB、Impala、Hive、Cloudera、Hue(使用)}
\cvline{框架}{Flask、Tornado、MapReduce 计算框架(熟悉)}
\cvline{计算机科学}{算法和数据结构(熟悉),计算机网络、编译原理和编程语言原理、计算机系统及硬件体系、计算机操作系统(了解)}
\cvline{文档撰写}{Markdown(熟悉),Word、LaTeX(简单了解)}
\cvline{外语}{英语(流畅地读,简单地听、说、写)}

\section{其它信息}
\cvline{Github}{https://github.com/winiex}
\cvline{About Me}{http://about.me/winiex}
\cvline{兴趣爱好}{音乐、阅读}

\end{document}
