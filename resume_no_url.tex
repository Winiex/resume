\documentclass[9pt]{ctexart}
\usepackage{hyperref}
\usepackage{fullpage}
\usepackage{amsmath}
\usepackage{amssymb}
\usepackage[usenames]{color}

\leftmargin=0.25in
\oddsidemargin=0.25in
\textwidth=6.0in
\topmargin=-0.25in
\textheight=9.25in

\raggedright

\pagenumbering{arabic}

\def\bull{\vrule height 0.8ex width .7ex depth -.1ex }
% DEFINITIONS FOR RESUME

\newenvironment{changemargin}[2]{%
    \begin{list}{}{%
            \setlength{\topsep}{0pt}%
            \setlength{\leftmargin}{#1}%
            \setlength{\rightmargin}{#2}%
            \setlength{\listparindent}{\parindent}%
        \setlength{\itemindent}{\parindent}%
            \setlength{\parsep}{\parskip}%
        }%
  \item[]}{\end{list}
      }

      \newcommand{\lineover}{
          \begin{changemargin}{-0.05in}{-0.05in}
              \vspace*{-8pt}
              \hrulefill \\
              \vspace*{-2pt}
          \end{changemargin}
      }

      \newcommand{\header}[1]{
          \begin{changemargin}{-0.5in}{-0.5in}
              \scshape{#1}\\
              \lineover
          \end{changemargin}
      }

      \newcommand{\contact}[4]{
          \begin{changemargin}{-0.5in}{-0.5in}
              \begin{center}
                  {\Large \scshape {#1}}\\ \smallskip
                  {#2}\\ \smallskip 
                  {#3}\\ \smallskip
                  {#4}\smallskip
              \end{center}
          \end{changemargin}
      }

      \newenvironment{body} {
          \vspace*{-16pt}
          \begin{changemargin}{-0.25in}{-0.5in}
          }	
      {\end{changemargin}
      }	

      \newcommand{\school}[4]{
          \textbf{#1} \hfill \emph{#2\\}
          #3\\ 
          #4\\
      }

      % END RESUME DEFINITIONS

      \begin{document}

      %%%%%%%%%%%%%%%%%%%%%%%%%%%%%%%%%%%%%%%%%%%%%%%%%%%%%%%%%%%%%%%%%%%%%%%%%%%%%%%%
      % Name
      \contact{聂伟琳}{电话号码:15271814355}{毕业学校:华中科技大学}{电邮:winiex.nie@gmail.com}


      %%%%%%%%%%%%%%%%%%%%%%%%%%%%%%%%%%%%%%%%%%%%%%%%%%%%%%%%%%%%%%%%%%%%%%%%%%%%%%%%
      % Objective
      \header{求职意向}

      \begin{body}
          \vspace{14pt}
          安卓应用程序开发相关的暑期实习职位,意向城市:深圳、北京、杭州
      \end{body}

      \smallskip


      %%%%%%%%%%%%%%%%%%%%%%%%%%%%%%%%%%%%%%%%%%%%%%%%%%%%%%%%%%%%%%%%%%%%%%%%%%%%%%%%
      % Education
      \header{教育经历}

      \begin{body}
          \vspace{14pt}
          \textbf{计算机科学与技术学士}{} \hfill \emph{2014 年毕业}{} \\
          \emph{华中科技大学,湖北,武汉}{} \\
          \medskip
      \end{body}

      \smallskip


      %%%%%%%%%%%%%%%%%%%%%%%%%%%%%%%%%%%%%%%%%%%%%%%%%%%%%%%%%%%%%%%%%%%%%%%%%%%%%%%%
      % Experience
      \header{工作经历}

      \begin{body}
          \vspace{14pt}
          \textbf{安卓应用程序开发(Part Time)}, \emph{武汉歪伯乐} \hfill \emph{2012 冬 - 现今}\\
          \vspace*{-4pt}
      \begin{itemize} \itemsep -0pt  % reduce space between items
              \item 负责公司“校园招聘”业务安卓客户端的开发和维护
              \item 完成应用最初始的原型设计,整个原型符合 Android 4.0 规范
              \item 和同事一起完成应用的基础框架,包括基本工具类、数据类型、网络异步处理框架
              \item 在该应用的基础框架之上完成了一些重要页面,并和设计、产品人员交流完善
              \item 后期根据用户反馈进行应用的修改和维护
              \item 产品主页为:http://www.ybole.com/xyzp
          \end{itemize}

          \textbf {安卓应用程序开发(实习)}, \emph{腾讯北京分公司} \hfill \emph{2011 暑期}\\
          \vspace*{-4pt}
      \begin{itemize} \itemsep -0pt
              \item 参与了地图部门安卓客户端实验性新功能的构思和实现
              \item 早期和实习生同事搭档完成了新特性的构想,并和组内同事一起讨论,完成产品设计方案和文档
              \item 后期进行特性的尝试性开发
          \end{itemize}
      \end{body}

      \smallskip


      %%%%%%%%%%%%%%%%%%%%%%%%%%%%%%%%%%%%%%%%%%%%%%%%%%%%%%%%%%%%%%%%%%%%%%%%%%%%%%%%
      % Experience
      \header{个人项目}

      \begin{body}
          \vspace{14pt}
          \textbf{Wanno}, \emph{基于 KNN 模型的图像自动标注、检索系统} \hfill \emph{进行中}\\
          \vspace*{-4pt}
      \begin{itemize} \itemsep -0pt  % reduce space between items
              \item 该项目为研究 Computer Vision 相关问题所涉及之项目,主要内容为完成一个简单的图像自动标注、检索系统
              \item 完成相关领域(机器学习、图像处理、计算机视觉)的基础知识学习
              \item 决定应用展示形态,完成技术方案选型
              \item 建立训练数据库,使用 Python 完成简单的爬虫的编写
              \item 利用 Python 上处理 Computer Vision 任务的 SimpleCV 完成图像的特征提取和 KNN 模型的构建、训练
              \item 完成简单的用于展示的 Web 应用
              \item 项目源码地址:https://github.com/Winiex/wanno
          \end{itemize}

          \newpage

          \textbf{Life TraXer}, \emph{LBS 类型应用 Demo} \hfill \emph{2011 冬}\\
          \vspace*{-4pt}
      \begin{itemize} \itemsep -0pt  % reduce space between items
              \item 该项目为参加学校范围内的安卓开发竞赛而开发,整个项目的前后端都由本人完成
              \item 完成该应用的产品构思,技术选型
              \item 开发简单的服务端程序,主要使用到的技术为 Java Servlet、MySql
              \item 开发该应用的安卓客户端
              \item 项目源码地址:https://github.com/Winiex/LifeTraXer
          \end{itemize}

          \textbf {CodePad}, \emph{安卓平台源代码阅读器} \hfill \emph{2011}\\
          \vspace*{-4pt}
      \begin{itemize} \itemsep -0pt
              \item 该项目基于废弃的开源项目进行改进、完善
              \item 早期熟悉源代码,并思考、设计新特性
              \item 实现新特性
              \item 项目源码地址:https://github.com/Winiex/codepad
          \end{itemize}

          \vspace*{-4pt}
          \textbf {wnx-regngin}, \emph{一个原始的正则引擎} \hfill \emph{2012}\\
      \begin{itemize} \itemsep -0pt
              \item 该项目的初始目的是为了实现一个非常简单的正则引擎,进而掌握初级的编译原理的相关知识,主要是词法分析部分
              \item 查阅资料,设计技术实现方案,最终选用了 Thompson 算法
              \item 用 C 语言实现相关算法
              \item 项目源码地址:https://github.com/Winiex/wnx-regngin
          \end{itemize}

          \vspace*{-4pt}
          \textbf {其它更多的个人项目源代码请浏览我的 Github 主页:https://github.com/Winiex}
      \end{body}

      \smallskip

      %%%%%%%%%%%%%%%%%%%%%%%%%%%%%%%%%%%%%%%%%%%%%%%%%%%%%%%%%%%%%%%%%%%%%%%%%%%%%%%%
      % Skills
      \header{个人技能}

      \begin{body}
          \vspace{14pt}
          \emph{\textbf{编程语言:}}{} Java, C, Python (熟练度递减)\\
          \medskip
          \emph{\textbf{使用的操作系统:}}{} Arch Linux, Ubuntu, Windows \\
          \medskip
          \emph{\textbf{框架、工具和平台:}}{} Android, Git, Vim, Eclipse, \LaTeX, MySql, Sqlite, Ruby on Rails, Github \\
          \medskip
          \emph{\textbf{计算机科学:}}{} 算法和数据结构,计算机网络,编译原理和编程语言原理,计算机系统及硬件体系,计算机操作系统,机器学习,图像处理 (除算法、数据结构和编译原理为做过小型实验,还有机器学习和图像处理为毕业设计课题外,其它均为了解基础知识的程度)

      \end{body}

      \smallskip


      %%%%%%%%%%%%%%%%%%%%%%%%%%%%%%%%%%%%%%%%%%%%%%%%%%%%%%%%%%%%%%%%%%%%%%%%%%%%%%%%
      % Leadership
      \header{社团活动}

      \begin{body}
          \vspace{14pt}
          \textbf{产品线副主席},华中科技大学腾讯创新俱乐部 \hfill {} \emph{2010 夏 - 2012 冬}\\
          \textbf{创办人、组长},华科技术宅(http://www.douban.com/group/husttgeek/) \hfill {} \emph{2012 冬}\\
      \end{body}

      \smallskip


      %%%%%%%%%%%%%%%%%%%%%%%%%%%%%%%%%%%%%%%%%%%%%%%%%%%%%%%%%%%%%%%%%%%%%%%%%%%%%%%%
      % Awards and Honors
      \header{荣誉奖项}

      \begin{body}
          \vspace{14pt}
          \textbf{人民奖学金(两次)}, 华中科技大学\\
          \smallskip
      \end{body}

      \header{其他信息}

      \begin{body}
          \vspace{14pt}
          \textbf{个人博客}:{http://bachiscoding.com}\\
          \smallskip
      \end{body}

      \end{document}
